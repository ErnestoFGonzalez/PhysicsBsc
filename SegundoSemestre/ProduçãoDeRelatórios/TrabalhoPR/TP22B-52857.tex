\documentclass[11pt]{beamer}

\usetheme{Berlin}
\usecolortheme{beaver}

%----------- PACOTES ---------%
\usepackage[latin1,utf8]{inputenc}
\usepackage[portuguese]{babel}
\usepackage{amssymb}
\usepackage{eurosym}
\usepackage{xcolor,colortbl}
\usepackage{adjustbox}
\usepackage{graphicx}
\usepackage{pgfplots} 
\usepackage{pgf-pie}
 
\pgfplotsset{width=10cm,compat=1.9}
%-----------------------------%
%----------- CORES ----------%
\definecolor{aquamarine}{rgb}{0.5, 1.0, 0.83}
\definecolor{airforceblue}{rgb}{0.36, 0.54, 0.66}
%-----------------------------%

\begin{document}
%----------- TÍTULO ----------%
\title{Finanças Pessoais\\
Produção de Relatórios}
\author{Ernesto Fernandes González\\
nº 52857 \quad TP 22 \\
email: efergo9@gmail.com\\
Licenciatura em Física}
\date{2018/2019}
\maketitle
%-----------------------------%

\begin{frame}
\begin{table} []
	\centering
	\begin{adjustbox}{width=\textwidth}
\begin{tabular}{|>{\columncolor{airforceblue}}c|c|c|c|c|c|c|c|} \hline
\rowcolor{airforceblue} \text{RESUMO} & \text{Janeiro} & \text{Fevereiro} & \text{Março} & \text{Abril} & \text{Maio} & \text{Junho} & \text{TOTAL} \\ \hline
\text{Rendimentos} & 534 & 534 & 534 & 534 & 534 & 534 & 3204.00 \\ \hline
\text{Gastos} & 514.64 & 506.23 & 497.85 & 462.03 & 329 & 383.46 & 2693.21 \\ \hline
\text{Saldos do mês} & 19.36 & 27.77 & 36.15 & 71.97 & 205 & 150.54 & 510.79 \\ \hline
\text{Saldo acumulado} & 19.36 & 47.13 & 83.28 & 155.25 & 360.25 & 510.79 & 510.79 \\ \hline
\end{tabular}
	\end{adjustbox}
\caption{Rendimento e gastos totais mensais e balanço financeiro. Os valores são apresentados em euros(\euro)}
\end{table}
\end{frame}

\begin{frame}
\begin{figure} []
	\centering
	\begin{adjustbox}{width=\textwidth}
    \begin{tikzpicture}[scale=.2]
\begin{axis}[
    title= Rendimentos versus Gastos,
    width=7cm,
    height=5cm,
    enlargelimits=0.15,
    legend style={at={(0.5,-0.15)},
    anchor=north,legend columns=-1},
    ylabel={Euros},
    ymin=0, ymax= 600,
    symbolic x coords={Janeiro, Fevereiro, Marco, Abril, Maio, Junho},
    xtick=data,
    x tick label style={rotate=45,anchor=east},
    nodes near coords align={vertical},
    ]
\addplot[ybar, nodes near coords, fill=black!10] 
    coordinates {(Janeiro,534) (Fevereiro,534) (Marco,534) (Abril,534) (Maio,534) (Junho,534)};
\addplot[draw=blue,ultra thick] 
    coordinates {(Janeiro,514.64) (Fevereiro,506.23) (Marco,497.85) (Abril,462.03) (Maio,329) (Junho,383.46)};
	\legend{$Rendimentos$,$Gastos$}
\end{axis}
    \end{tikzpicture}
    \end{adjustbox}
    \caption{Rendimentos versus Gastos}
\end{figure}
\end{frame}

\begin{frame}
\begin{table}
$$\begin{array}{|>{\columncolor{airforceblue}}c|c|} \hline
\rowcolor{airforceblue} \text{As contas} & \text{Total} \\ \hline
\text{Rendimentos} & 3204 \\ \hline
\text{Habitação} & 1307.51 \\ \hline
\text{Saúde} & 104.25 \\ \hline
\text{Transportes} & 318.25 \\ \hline
\text{Despesas pessoais} & 93.26 \\ \hline
\text{Lazer} & 64.85 \\ \hline
\text{Escola} & 584.20 \\ \hline
\end{array}$$
\caption{Rendimentos e gastos totais, divididos por setores. Os valores são apresentados em euros(\euro)}
\end{table}
\end{frame}

\begin{frame}
\begin{figure} []
\begin{tikzpicture}
\pie [rotate = 180, text = legend, explode = 0.2, color = {blue,green,red,yellow,orange,purple}]
    {58.09/Habitação,
     3.87/Saúde, 11.82/Transportes, 3.46/Despesas Pessoais, 2.41/Lazer, 20.35/Ensino}
\end{tikzpicture}
\caption{Despesas em Percentagem}
\end{figure}
\end{frame}

\begin{frame}
\frametitle{Preenchimento da tabela Planeamento}
No preenchimento da tabela planeamento seguiu-se o princípio da divis
ão de gastos por áreas (i.e habitação, saúde, educação), que por sua vez se dividiam em atividades ou produtos(i.e medicamentos, médico).

\center{\includegraphics[scale=0.1]{planeamento}}
\end{frame}

\begin{frame}
\includegraphics[scale=0.4]{personalfinance}
\footnote{Imagem retirada de https://financialwellness.org/wp-content/uploads/2017/12/Personal-Finance-Tips-Piggy-Bank-Coins-Personal-Finances-in-2018-SS-Feature.jpg}
\end{frame}

\end{document}