\documentclass[12pt, twoside, a4paper]{article}

\usepackage{graphicx}
\usepackage{color}

\usepackage{amsmath}
\usepackage{amssymb}
\usepackage{amsthm}

\usepackage[portuguese]{babel}
\usepackage[utf8]{inputenc}
\usepackage[T1]{fontenc}
\usepackage{enumerate}

\begin{document}

%%%%%%%%%%%%%%Título%%%%%%%%%%%%%%%%%%
\title{Challenge 1}
\author{João Cordeiro 53688\\José Lopes 52878\\João Olívia 52875\\Ernesto González 52857}
\date{2019-2020}
\maketitle
%%%%%%%%%%%%%%%%%%%%%%%%%%%%%%%%%%%%%%

%%%%%%%%%%%%%%%%%%%%%%%%%%%%%%%%%%%%%%
\section{Problema 1}
Considere-se uma expansão adiabática reversível do gás ideal entre um estado inicial 1 e um estado final 2. Mostre que:
\begin{enumerate} [a)]
  \item $W_{1\rightarrow 2}=-\frac{3}{2}NK_B(T_1-T_2)$ \\
    \begin{proof}
      Tem-se que, para um processo adiabático $$dQ=0, $$ então pela 1ª Lei da Termodinâmica vem $$dU = dQ+ dW=dW. $$\\
      Por outro lado, para o gás ideal, $$dU=C_VdT,$$ com $C_V =\frac{3}{2}NK_B.$\\
      Então obtemos que $$dW=C_VdT=\frac{3}{2}NK_BdT. $$\\
      Integrando, entre 1 e 2
      $$W_{1\rightarrow2}= \int_{T_1}^{T_2}\frac{3}{2}NK_BdT=\frac{3}{2}NK_B\int_{T_1}^{T_2}dT $$
      \begin{equation}
        W_{1\rightarrow2}=\frac{3}{2}NK_B(T_2-T_1)=-\frac{3}{2}NK_B(T_1-T_2)
      \end{equation}
    \end{proof}

  \item $W_{1\rightarrow 2}=\frac{p_2V_2-p_1V_1}{\gamma-1}$ \\
    \begin{proof}
      De (1) $$W_{1\rightarrow2}=-\frac{3}{2}NK_BT_1+\frac{3}{2}NK_BT_2 $$
      Recordando que para o gás ideal, $pV=NK_BT$ vem
      $$W_{1\rightarrow2}=\frac{3}{2}(p_2V_2-p_1V_1)=\frac{p_2V_2-p_1V_1}{\frac{2}{3}}=\frac{p_2V_2-p_1V_1}{\frac{5}{3}-1}.$$
      Sabendo que o índice adiabático é $\gamma=\frac{C_p}{C_V}=\frac{5}{3}$ e fácil ver que
      \begin{equation}
        W_{1\rightarrow2}=\frac{p_2V_2-p_1V_1}{\gamma-1}.
      \end{equation}
    \end{proof}

  \item $W_{1\rightarrow 2}= \frac{p_2V_2}{\gamma-1}\left[1-\left(\frac{p_1}{p_2}\right)^{\frac{\gamma-1}{\gamma}}\right]$
    \begin{proof}
      Para uma expansão adiabática do gás ideal reversível obteve-se em aula o seguinte resultado:
      \begin{equation}
        \frac{p_2V_2}{p_1V_1}=\left(\frac{V_1}{V_2}\right)^{\gamma-1}
    \end{equation}
    De (3) vem $$p_1V_1=p_2V_2\left(\frac{V_2}{V_1}\right)^{\gamma-1}.$$ Substituindo em (2),
    \begin{equation}
      W_{1\rightarrow2}=\frac{p_2V_2-p_2V_2\left(\frac{V_2}{V_1}\right)^{\gamma-1}}{\gamma-1}=\frac{p_2V_2}{\gamma-1}\left[1-\left(\frac{V_2}{V_1}\right)^{\gamma-1}\right].
    \end{equation}
    Mais uma vez, de (3) vem
    $$\frac{p_2}{p_1}=\left(\frac{V_1}{V_2}\right)^{\gamma} \iff \frac{V_1}{V_2}=\left(\frac{p_2}{p_1}\right)^{\frac{1}{\gamma}} \iff \frac{V_2}{V_1}=\left(\frac{p_1}{p_2}\right)^{\frac{1}{\gamma}}.$$
    Substituindo em (4), ficamos com
    $$W_{1\rightarrow2}=\frac{p_2V_2}{\gamma-1}\left[1-\left(\frac{p_1}{p_2}\right)^{\frac{\gamma-1}{\gamma}}\right] $$
    \end{proof}
\end{enumerate}

\section{Problema 2}
Partindo de $U=U(V,p)$, mostre que:
$$C_p=\left[p\left(\frac{\partial V}{\partial T}\right)_p+\left(\frac{\partial U}{\partial T}\right)_p\right]. $$

\begin{proof}
  Como $U=U(V,p)$, pela regra da cadeia surge
  \begin{equation}
     dU = \left(\frac{\partial U}{\partial V}\right)_p dV + \left(\frac{\partial U}{\partial p}\right)_V dp.
  \end{equation}
  Pela 1ª Lei da Termodinâmica, sabemos que $dU=dQ+dW$ e $dQ=CdT$. Tratando-se de trabalho reversível, vem $dW=-pdV$. Substituíndo em (5) vem
  \begin{equation}
    dU=CdT-pdV.
  \end{equation}
  Por (5) e (6),
  $$CdT-pdV=\left(\frac{\partial U}{\partial V}\right)_p dV+\left(\frac{\partial U}{\partial p}\right)_V dp $$
  Tomando $V=V(T,p)$, da regra da cadeia surge
  \begin{equation}
    dV=\left(\frac{\partial V}{\partial T}\right)_p dT + \left(\frac{\partial V}{\partial p}\right)_T dp.
  \end{equation}
  $$CdT-p\left[\left(\frac{\partial V}{\partial T}\right)_p dT + \left(\frac{\partial V}{\partial p}\right)_T + dp\right]=\left(\frac{\partial U}{\partial V}\right)_p dV + \left(\frac{\partial U}{\partial p}\right)_V dp $$
  $$CdT -p\left(\frac{\partial V}{\partial T}\right)_p dT - p\left(\frac{\partial V}{\partial p}\right)_T dp =\left(\frac{\partial U}{\partial V}\right)_p dV + \left(\frac{\partial U}{\partial p}\right)_v dp $$
  Se $p$ for constante, vem $dp = 0$, e portanto,
  $$CdT-p\left(\frac{\partial V}{\partial T}\right)_p dT = \left(\frac{\partial U}{\partial V}\right)_p dV. $$
  Por (7) vem,
  $$CdT-p\left(\frac{\partial V}{\partial T}\right)_p dT = \left(\frac{\partial U}{\partial V}\right)_p \left(\frac{\partial V}{\partial T}\right)_p dT. $$
  Pela regra da cadeia,
  $$CdT - p\left(\frac{\partial V}{\partial T}\right)_p dT = \left(\frac{\partial U}{\partial T}\right)_p dT $$
  $$C_p = \left[p\left(\frac{\partial V}{\partial T}\right)_p + \left(\frac{\partial U}{\partial T}\right)\right]$$
  em que $C_p$ é a capacidade calorífica a pressão constante.
\end{proof}

\section{Problema 3}
Numa canção de paródia à ciência Flanders e Swan sumarizam a primeira lei da termodinâmica dizendo que \textit{heat is work and work is heat}. Comente.\\
Tendo em conta que tanto trabalho ("work") como calor ("heat") são entidades energéticas, então podemos afirmar que são "interchangeable" pelo que são manifestações diferentes da mesma entidade - energia. Segundo Einstein, até massa podia entrar na letra da paródia.


\end{document}
