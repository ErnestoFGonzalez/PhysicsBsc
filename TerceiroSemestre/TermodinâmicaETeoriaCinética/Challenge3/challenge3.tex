\documentclass[12pt, twoside, a4paper]{article}

\usepackage{graphicx}
\usepackage{color}

\usepackage{amsmath}
\usepackage{amssymb}
\usepackage{amsthm}

\usepackage[portuguese]{babel}
\usepackage[utf8]{inputenc}
\usepackage[T1]{fontenc}
\usepackage{enumerate}

\begin{document}

%%%%%%%%%%%%%%Título%%%%%%%%%%%%%%%%%%
\title{Challenge 3}
\author{João Cordeiro 53688\\José Lopes 52878\\João Olívia 52875\\Ernesto González 52857}
\date{2019-2020}
\maketitle
%%%%%%%%%%%%%%%%%%%%%%%%%%%%%%%%%%%%%%

%%%%%%%%%%%%%%%%%%%%%%%%%%%%%%%%%%%%%%
\section{Problema 1}
Considere-se uma expansão adiabática reversível do gás ideal entre um estado inicial 1 e um estado final 2. Mostre que:
\begin{enumerate}[1.]
  \item Calcule a capacidade calorífica a volume constante,
  \begin{equation}
    C_V=\left(\frac{\partial U}{\partial T}\right)_V,
  \end{equation}
  e a capacidade calorífica a pressão constante,
  \begin{equation}
    C_p=\left(\frac{\partial H}{\partial T}\right)_p,
  \end{equation}
  e verifique que $C_p>C_V$.
  \begin{proof}
    No gás ideal temos
    \begin{equation}
      U = \frac{3}{2}Nk_BT
    \end{equation}
    Diferenciando em ordem a $T$ vem
    \begin{equation}
      C_V=\left(\frac{\partial U}{\partial T}\right)_V = \frac{3}{2}Nk_B
    \end{equation}
  \end{proof}
  \begin{proof}
    Temos
    \begin{equation}
      H = U+ pV = \frac{3}{2}Nk_BT+pV
    \end{equation}
    onde fizemos uso de (3).
    Da lei dos gases
    \begin{equation}
      pV=Nk_BT
    \end{equation}
    Logo (5) passa a
    \begin{equation}
      H = \frac{5}{2}Nk_BT
    \end{equation}
    Diferenciando (7) em ordem a $T$ vem
    \begin{equation}
      C_p = \left(\frac{\partial H}{\partial T}\right)_p = \frac{5}{2}Nk_B
    \end{equation}
  \end{proof}
  Assim
  \begin{equation}
    \frac{5}{2}Nk_B = C_p > C_V = \frac{3}{2}Nk_B
  \end{equation}
  \item O resultado que obteve é geral? Ou seja, é verdade que, independentemente do sistema físico que estivermos a estudar, $C_p > C_V$? Justifique a sua resposta.
  Temos
  \begin{equation}
    dU = dQ + dW
  \end{equation}
  e
  \begin{equation}
    U = \frac{3}{2}Nk_BT
  \end{equation}
  Logo $U=U(T)$. Temos também
  \begin{equation}
    H=U+pV
  \end{equation}
  Logo $H=H(T,p)$, e por (11) $H=H(U,p)$.\\
  Daqui surge
    \begin{equation}
      C_V < C_p
    \end{equation}
\end{enumerate}

\section{Problema 2}
Considere um gás idela formado por $N$ partículas com energia interna $U=\frac{3}{2}Nk_BT$.
\begin{enumerate}[1.]
  \item Considere que N é constante e use a equação fundamental da termodinâmica para mostrar que num processo entre um estado inicial 1 e um estado final 2
  \begin{equation}
    \Delta S=\frac{3}{2}Nk_B\,ln(\frac{U_2}{U_1})+Nk_B\,ln(\frac{V_2}{V_1})
  \end{equation}
  \begin{proof}
    Da equação fundamental da termodinâmica
    \begin{equation}
      dU=TdS-pdV+\mu dN.
    \end{equation}
    A $N$ constante, $dN=0$ e a equação fundamental passa a
    \begin{equation}
      dU=TdS-pdV \iff dS=\frac{1}{T}dU+\frac{p}{T}dV
    \end{equation}
    Da lei dos gases vem
    \begin{equation}
      pV=Nk_BT \iff \frac{p}{T} = Nk_B\frac{1}{V}.
    \end{equation}
    Temos que $U=\frac{3}{2}Nk_BT$, logo
    \begin{equation}
      \frac{1}{T}=\frac{3}{2}Nk_B\frac{1}{U}.
    \end{equation}
    Assim, de (12) temos
    \begin{equation}
      dS = \frac{3}{2}Nk_B\frac{1}{U}dU+Nk_B\frac{1}{V}dV
    \end{equation}
    e integrando
    \begin{equation}
      \begin{split}
      \Delta S&=\frac{3}{2}Nk_B \int_{U_1}^{U_2}\frac{1}{U}dU + Nk_B \int_{V_1}^{V_2}\frac{1}{V}dV \iff \\
      \iff \Delta S&=\frac{3}{2}Nk_B\,ln(\frac{U_2}{U_1})+Nk_B\,ln(\frac{V_2}{V_1})
      \end{split}
    \end{equation}
  \end{proof}
  \item
  Considere que $N$ é constate e partindo de $S=S(T,V,N)$ mostre que num processo entre um estado inicial 1 e um estado final 2
  \begin{equation}
    \Delta S=\frac{3}{2}Nk_B\,ln(\frac{U_2}{U_1})+Nk_B\,ln(\frac{V_2}{V_1})
  \end{equation}
  \begin{proof}
    Considerando $S=S(T,V,N)$ vem
    \begin{equation}
      dS=\left(\frac{\partial S}{\partial T}\right)dT+\left(\frac{\partial S}{\partial V}\right)dV+\left(\frac{\partial S}{\partial N}\right)dN
    \end{equation}
    Da equação fundamental da termodinâmica
    \begin{equation}
      \begin{split}
      dU &= TdS-pdV+\mu dN\\
      dS &= \frac{1}{T}dU+ \frac{p}{T}dV - \frac{\mu}{T}dN.
      \end{split}
    \end{equation}
    Sabemos que
    \begin{equation}
      \begin{split}
        U = \frac{3}{2}Nk_BT \iff \frac{1}{T}=\frac{3}{2}Nk_B\frac{1}{U}
      \end{split}
    \end{equation}
    Analogamente
    \begin{equation}
      \begin{split}
        dU = \frac{3}{2}Nk_B\,dT \iff dT=\frac{1}{\frac{3}{2}Nk_B}dU
      \end{split}
    \end{equation}
    Como $dN=0$, de (23) vem
    \begin{equation}
      dS=\left(\frac{3}{2}Nk_B\right)^2 \frac{1}{U}dT+Nk_B\frac{1}{U}dU
    \end{equation}
    e
    \begin{equation}
      \left(\frac{\partial S}{\partial T}\right)=\left(\frac{3}{2}Nk_B\right)^2 \frac{1}{U} \quad \left(\frac{\partial S}{\partial V}\right)=Nk_B\frac{1}{V}
    \end{equation}
    Então
    \begin{equation}
      \begin{split}
        dS &= \left(\frac{3}{2}Nk_B\right)^2\frac{1}{U}dT+Nk_B\frac{1}{V}dV \\
        dS &= \frac{3}{2}Nk_B\frac{1}{U}dU+Nk_B\frac{1}{V}dV\\
        \Delta S&=\frac{3}{2}Nk_B \int_1^2 \frac{1}{U}dU + Nk_B \int_1^2 \frac{1}{V}dV \\
        \Delta S&=\frac{3}{2}Nk_B\,ln\left(\frac{U_2}{U_1}\right)+ Nk_B\,ln \left(\frac{V_2}{V_1}\right)
      \end{split}
    \end{equation}
  \end{proof}

  \item Partindo de $s=s(u,v)$ em que $s$, $u$ e $v$ são quantidades parciais molares, mostre que
  \begin{equation}
    \Delta s= \frac{3}{2}k_B\,ln\left(\frac{u_2}{u_1}\right)+k_b\,ln\left(\frac{v_2}{v_1}\right)
  \end{equation}
  Considerando $s=s(u,v)$ temos
  \begin{equation}
    ds=\left(\frac{\partial s}{\partial u}\right)du + \left(\frac{\partial s}{\partial v}\right)dv
  \end{equation}
  Vimos que
  \begin{equation}
    \left(\frac{dU}{dS}\right)_V dU = T
  \end{equation}
  Como estamos a trabalhar a $N$ constante e
  \begin{equation}
    u=\frac{U}{N}
  \end{equation}
  Então
  \begin{equation}
    ds=\frac{1}{T}du + \left(\frac{ds}{dv}\right)dv
  \end{equation}
  Da lei dos gases
  \begin{equation}
      pV=Nk_BT \iff p=\frac{Nk_B}{V}T
  \end{equation}
  e dividindo por $N$
  \begin{equation}
      pv=k_BT \iff p=\frac{k_B}{v}T \iff \frac{dp}{dT}=k_B\frac{1}{V}
  \end{equation}
  Assim vem
  \begin{equation}
    \left(\frac{\partial s}{\partial v}\right)_u = \left(\frac{\partial s}{\partial v}\right)_T = \left(\frac{\partial p}{\partial T}\right)_s = k_B \frac{1}{v}
  \end{equation}
  Finalmente substituíndo em (33),
  \begin{equation}
    \begin{split}
      \Delta s &= \frac{3}{2}k_B \int_1^2 \frac{1}{u}dU + k_B\int_1^2\frac{1}{v}dv \iff \\
      \iff \Delta s &=\frac{3}{2}k_B\,ln\left(\frac{u_2}{u_1}\right)+ k_B\,ln\left(\frac{v_2}{v_1}\right)
    \end{split}
  \end{equation}
\end{enumerate}





\end{document}
